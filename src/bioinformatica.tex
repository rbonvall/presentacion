\frame
{
\frametitle{\oran{Bioinformática}}

\begin{columns}
\column{0.55\textwidth}
\begin{itemize}
\item FONDEF G09i1007,  \emph{Desarrollo y aplicaci\'{o}n de herramientas de gen\'{o}mica e ingenier\'{i}a gen\'{e}tica para potenciar el fitomejoramiento de vides de mesa}
\item Participantes:
	\begin{itemize}
	\item Coordinador: INIA
	\item Equipo bioinform\'{a}tico: 
		\begin{itemize}
		\item Dr. Alexander Zamyatnin (CTI-HPC, UTFSM)
		\item Dr. Francisco González (UVM)
		\end{itemize}
	\end{itemize}
\end{itemize}
\column{0.45\textwidth}
\begin{center}
\pgfuseimage{uva}
\end{center}
\end{columns}
}
% El objetivo del presente estudio es la identificación de nuevos oligopéptidos antimicrobianos con estudio funcional de las secuencias de proteínas de uva de Vitis vinifera.

%dedicada a mejorar la resistencia natural de los animales y las plantas contra hongos, patógenos y enfermedades.
