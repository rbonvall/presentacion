\frame
{
\frametitle{\oran{Capacitación y Difusión de HPC}}

%Apertura hacia nuevas áreas de investigación. Organización de eventos de carácter mundial, relacionados con HPC.

\begin{columns}
\column{0.6\textwidth}
\begin{itemize}
\item Eventos como:
\begin{itemize} 
\item  EPIKH (Exchange Programme to advance e-Infrastructure Know-How) $\rightarrow$ grid computing
\item SCAT (Scientific Computing-Advanced Training) y PASI (Panamerican Advanced Studies Institute) $\rightarrow$ computaci\'{o}n cient\'{i}fica y HPC
\end{itemize} 
garantizan 
la experticia generada, tanto a nivel organizativo como colaborativo. 
\item Instancia importante para capacitación de integrantes dentro del CTI-HPC. 
\end{itemize}
\column{0.4\textwidth}
\pgfuseimage{evento}
\end{columns}
}


\frame
{
\frametitle{\oran{Capacitación y Difusión de HPC}}
\framesubtitle{PASI 2011: \emph{Scientific Computing in the Americas: the challenge of massive parallelism}.}

\begin{columns}
\column{0.6\textwidth}
\begin{itemize}
\item  Escuela de Verano organizada por Boston University y  UTFSM
\item 3–14 Enero  2011, Valpara\'{i}so, Chile
\item Financiamiento de la National Science Foundation (NSF):
        \begin{itemize}
        \item Financiamento de expositores y becas para 37 alumnos de EEUU y
Latinoamérica
        \end{itemize}
\item Algunos resultados: CUDA Teaching Center, Colaboración con el SHOA.
\end{itemize}
\column{0.4\textwidth}
\pgfuseimage{pasi}
\end{columns}

}




\frame
{
\frametitle{\oran{Capacitación y Difusión de HPC}}

\begin{columns}
\column{0.6\textwidth}
\begin{itemize}
\item Diversos cursos durante el año 2011:
% Cursos de programación para instituciones nacionales, Procesamiento de Imágenes Satelitales y desarrollo de aplicaciónes de Alto Desempeño sobre CUDA, permite la difusión de nuestras actividades.
\begin{itemize}
\item Julio 2011, $1^{er}$ Curso de Proc. de Imág. Satelitales
\item Agosto 2011, Curso de GPU Computing
\item Noviembre 2011, $2^{do}$ Curso de Proc. de Imág. Satelitales
\item Diciembre 2011, Tutorial Grid Computing UTFSM-UFRO
\end{itemize}
\end{itemize}
\column{0.4\textwidth}
\pgfuseimage{curso}
\end{columns}

}



